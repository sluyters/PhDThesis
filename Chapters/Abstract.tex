\chapter*{Abstract}
% Context
Radar sensors combine a small form factor and low power consumption with the ability to sense gestures through opaque surfaces and under unfavorable lighting conditions, all while preserving user privacy. 
%
These unique advantages could make them a credible and useful alternative to other types of sensors, such as cameras or inertial sensors, for gesture recognition.

% Limitations of radar
However, research on radar-based gestural interfaces is in its infancy and many challenges that hinder their seamless integration and widespread adoption still must be solved.
%
In particular, the lack of standardization translates into a plethora of custom radar sensors and techniques, preventing efficient collaboration between developers. 
%
In addition, radar signals can be noisy and complex to analyze, and do not transpose well from one radar to another.

% Solution
This thesis investigates and advances the state of radar-based gesture interaction in two stages.
%
First, it explores the use of radar sensors for gesture recognition by reporting results from a targeted literature review and two systematic literature reviews, unveiling a large variety of radar sensors, gesture sets, and gesture recognition techniques, as well as the many challenges of real-time gesture recognition.
%
Second, it provides tools and methods that facilitate the development of highly usable radar-based gesture interfaces by bridging the gap between researchers and practitioners.
%
In particular, it introduces \ql, a modular framework for gesture recognition that acts as an intermediate layer between (radar) sensors and gesture-based applications and facilitates the performance and efficiency evaluation of gesture recognizers.
%
Additionally, it proposes a user-centered development method for gesture-based applications and applies it to a multimedia application.
%
Finally, it introduces and evaluates a new gesture recognition pipeline that implements advanced full-wave electromagnetic modeling and inversion to retrieve physical characteristics of gestures that are independent of the source, antennas, and radar-hand interactions.
\\[10pt]
\textbf{Keywords:} gesture recognition, radar, development environment.
